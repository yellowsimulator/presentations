
% !TeX spellcheck = en_GB
\documentclass{beamer}
\usepackage[utf8]{inputenc}
 \usetheme{antibes}
 \usepackage{movie15}
 \usepackage{graphicx}
 \usepackage{hyperref}
 \usepackage{tikz}
 \usetikzlibrary{shapes.geometric, arrows}
 \tikzstyle{startstop} = [rectangle, rounded corners, minimum width=3cm, minimum height=1cm,text centered, draw=blue, fill=blue!15]
 \tikzstyle{arrow} = [thick,->,>=stealth]
 \hypersetup{
 	colorlinks=true,
 	linkcolor=blue,
 	filecolor=magenta,      
 	urlcolor=cyan,
 }
 
 \urlstyle{same}
 \title[\textcolor{orange}{Karsten Moholt Digital}] %optional
 {Finite element method for predictive maintenance}
 
 %\subtitle{An optimization probblem}
 
 \author[] % (optional, for multiple authors)
 {}
 
% \institute[VFU] % (optional)
% {
% 	\inst{1}%
% 	Faculty of Physics\\
% 	Very Famous University
% 	\and
% 	\inst{2}%
% 	Faculty of Chemistry\\
% 	Very Famous University
% }
 
 %\date[VLC 2013] % (optional)
 %{Very Large Conference, April 2013}
 
 \logo{\includegraphics[height=1.5cm]{km-logo.png}}
 
 
\begin{document}
 
\frame{\titlepage}
 
 
 %%%%%%%%%%%%%%%%%%%%%%%%%%%%%%%%%%%%%%%%%%%%%%%%%%%%%%%%%%%
 
 
 
 %%%%%%%%%%%%%%%%%%%%%%%%%%%%%%%%%%%%%%%%%%%%%%%%%%%%%%%%%%%%
 

%%%%%%%%%%%%%%%%%%%%%%%%%%%%%%%%%%%%%%%%%%%%%%%%%%%%%%


%%%%%%%%%%%%%%%%%%%%%%%%%%%%%%%%%%%%%%%%%%%%%%%%%%%%%%%%
\begin{frame}
	\frametitle{Finite element method for defect propagation}
\begin{figure}[H]
	\begin{center}
		\begin{tikzpicture}
\node (machine) [startstop] {\textcolor{blue}{Machine}};
\node (mesh) [startstop, below of=machine, yshift=-0.5cm] {\textcolor{blue}{Mesh}};
\node (solver) [startstop, below of=mesh, yshift=-0.5cm] {\textcolor{orange}{Finite element solver}};
\node (prediction) [startstop, below of=solver, yshift=-0.5cm] {\textcolor{blue}{Prediction}};

%%% drawing %%%%%%%%%%%%%%%%%%%%%%%
\draw [arrow] (machine) -- (mesh);
\draw [arrow] (mesh) -- (solver);
\draw [arrow] (solver) -- (prediction);
		\end{tikzpicture}
	\end{center}

\end{figure}
\end{frame}
%%%%%%%%%%%%%%%%%%%%%%%%%%%%%%%%%%%%%%%%%%%%%%%%%%%%%%%%


\begin{frame}
	\frametitle{What is a finite element solver}

		
	\begin{itemize}
		\item A \textcolor{orange}{finite element solver} is a software that can be used for pressure, temperature and wear prediction.
		\item It takes a mesh as input
		\item It uses a mathematical model to make prediction based on the mesh
	\end{itemize}
\end{frame}
\end{document}

